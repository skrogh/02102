\section{Opgave 2}
\subsection{Beskrivelse og Design}
Der ønskes et program der kan simulere og give en grafisk repræsentation af en randomwalk i 2 dimensioner. \\
\\
Randomwalken skal starte i (0,0) i et 2n x 2n koordinatsystem hvor n er et tal som en bruger af programmet har specificeret. \\
Dette implementeres i et vores program ved at lave et 512x512pixel canvas der skaleres efter koordinatsystemets størrelse. 2 variable, der beskriver x og y for tegnepositionen, initialiseres og i et while loop bruges java util bibliotekets random klasse til at bestemme en tilfældig ændring af tegnepositionen hvorefter der tegnes en prik på tegnepositionen. Overskrider tegnepositionen koordinatsystemets grænser, afbrydes programmet.
\\
\\
Da det kan tage lang tid at tegne en randomwalk med tilhørende positionsoutput i konsollen, er der tilføjet en debug mode der skal tilslås for at få positionsændringer vist i konsollen.
\subsection{Programtest}
Først testes metoden, der får input fra brugeren:

\begin{lstlisting}
Type size for canvas: fisk
This is not an integer, using default of: 512
Type size of walk: -5
This is not a positive integer, using default of: 50
Debug mode? "y" for yes: n
Entering regular mode
Type "exit" to stop, anything else to draw a new walk: exit
Quitting
Program terminated
\end{lstlisting}
Det ses at både ord og negative tal forkastes. Skærm-outputtet til dette kan ses i figur~\ref{fig:walkM}

\begin{figure}[h!]
	\centering
	\includegraphics[width=0.5\textwidth]{walkM}
		\caption{Randomwalk med gitter på 101x101 og en opløsning på 512x512px}\label{fig:walkM}
\end{figure}

Dernæst testes debug print af position:
\begin{lstlisting}
Type size for canvas: 512
Type size of walk: 10
Debug mode? "y" for yes: y
Entering debug mode
Position: (0, 1)
Position: (-1, 1)
Position: (-1, 2)
... og mange flere ...
Position: (1, 10)
Position: (0, 10)
Position: (1, 10)
Type "exit" to stop, anything else to draw a new walk: exit
Quitting
Program terminated
\end{lstlisting}
Det ses at ``turisten''/''støvpartiklen''s position printes til konsollen. Skærm-outputtet til dette kan ses i figur~\ref{fig:walkS}

\begin{figure}[h!]
	\centering
	\includegraphics[width=0.5\textwidth]{walkS}
		\caption{Randomwalk med gitter på 21x21 og en opløsning på 512x512px}\label{fig:walkS}
\end{figure}

Til slut testes et gigantisk randomwalk, billed størrelsen er 4500x4500px og der gåes på et gitter på 2001x2001. Resultatet er i figur~\ref{fig:walkH}

\begin{figure}[h!]
	\centering
	\includegraphics[width=0.9\textwidth]{walkH}
		\caption{Randomwalk med gitter på 2001x2001 og en opløsning på 4500x4500px}\label{fig:walkH}
\end{figure}
