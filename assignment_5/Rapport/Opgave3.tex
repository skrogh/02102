\section{Opgave 3}
\subsection{Specifikation}
Der skal skrives et program \emph{cudb}, en database med studerende. For hver
studerende skal følgende gemmes:
\begin{itemize}
	\item Navn, bestående af max fire tegn
	\item Startår, et tal mellem 2009 og 2040, begge inklusive.
	\item Start semester: 0 for forår, 1 for efterår
	\item Karaktergennemsnit, et tal mellem 0 og 255, begge inklusive.
\end{itemize}

For at gemme dette kompakt, gemmes alt data, på nær navnet, i en enkelt integer.
Datafrodelingen over de 14 første bits er således:
\begin{itemize}
	\item Bit $[0;~4]$: Startår 
	\item Bit $5$: Semester
	\item Bit $[6;~13]$: Karakter
\end{itemize}

Desuden benyttes bit 14 til paritets kontrol og bit 15 indikerer en fejl under
oprettelse af den studerende.

Programmet skal være i stand til at modtage et antal inputs fra brugeren, i
form af tal der skrives til konsollen:
\begin{itemize}
	\item 0: Slut: Programmet afsluttes og lukker.
	\item 1: List alle studerende: En liste med alle studerende kommer frem og
	gennemsnittet af alle deres karakterer udregnes. Studerende med korrupt data
	medrenes ikke.
	\item 2: Tilføj studerende: Brugeren indtaster data om en studerende og denne
	tilføjes til systemet.
\end{itemize}
Deuden tilføjes to funktioner
\begin{itemize}
	\item 3: Test database: en dummy-database oprettes og et antal studerene
	tilføjes, nogle med ``gyldig'' data, andre med ugyldig. Dette gøres for nememre
	at teste databasesystemet.
	\item 4: Korrupter studerende: Vender bit 0, så paritetsbitten ikke længere
	passer med dataen. Simulerer fejl under oplagring af data.
\end{itemize}

Databasen skal kunne indeholde $10000$ studerende

\subsection{Design}
En type, \emph{student\_t} er allerede oprettet.
Denne indeholder et \emph{char array} med størrelsen fem,
til navnet (4 tegn plus terminering) og en \emph{int} til at gemme data.

En ny type, \emph{database\_t} oprettes. Denne type indeholder en variabel, der
gemmer antallet af studerende i databasen, samt et array på $10000$ studerende.

\subsubsection{Funktioner til behandling af \emph{student\_t} typen}
\paragraph{student\_t* encodeStudent( char* name, int year, int semester, int
grade )} Funktion der retunerer en variabel af \emph{student\_t} ud fra
de givne inputs.
Der oprettes en midliertidig \emph{student\_t} variabel og alle felter i denne
sætes til 0 (eller tilsvarende).
Det tjekkes om alle inputs overholder restriktionerne:
\begin{itemize}
	\item \emph{name} skal være på mindre end 4 tegn ex. slut-tegn
	\item \emph{year} skal være mellem 2009 og 2040, begge inklusive
	\item \emph{semester} skal være enten 1 eller 0.
	\item \emph{grade} skal være mellem 0 og 255, begge inklusive
\end{itemize}
Alle felter der ikke opfylder dette sættes ikke i variablen der retunerers, og
er derfor 0, desuden sættes fejflaget, bit 15 højt, hvis dette sker.

\emph{year} skaleres, så 2009 svarer til 0 og 2040 til 31.
\emph{year}, \emph{semester}, og \emph{grade} får lagt ``bit-masker'' på sig med
\& operationen og bit-forskydes med operationen $<<$ de lægges nu ind i data
feltet med operationen $|=$. \& operationen er egenligt ikke nødvendig, men
medtages da det eferfølgende tal indikerer bit-bredden for tallet der gemmes.
Det er kun muligt at gøre dette, uden at ryde de relevate områder i
\emph{data}-bitsne da \emph{data} er sat til 0 fra start.

Paritetsbitten enkodes nu. Dette gøre ved at ``xor''e alle relevante bits i
\emph{data} sammen med \^\ operationen. Denne bit er således høj, hvis der er
et ulige antal 1'ere i de relevante \emph{data}-bits. Dette kan bruges til at vise
om et ulige antal bits har ændret sig under lagringen af dataen.
For mere info om paritet læs: \url{http://en.wikipedia.org/wiki/Parity_bit}

\paragraph{int decodeStudentParity( student\_t* student )} retunerer 1, hvis der
er uverenstemmelse mellem paritetsbitten og resten af dataen i for den pågældende
studrende.

\paragraph{char* decodeStudentName( student\_t* student )} giver en pointer til
navnet på den pågældende studerende. Egenligt ikke en nødvenighed at have denne
funktion, men det skaber en vis symetri i resten af koden.

\paragraph{int decodeStudentYear( student\_t* student )} og resten af decode
funktionerne finder det korrekte sted i \emph{data} og benytter bit forskydning og maskering
til at retunere de originale værdier. for \emph{year} skaleres, så 0 bliver til
2009, 31 til 2040.

\paragraph{void printStudent( student\_t* student )} printer info om den givne
studerende til terminalen. Hvis der er uoverenstemmelse mellem data og
paritetsbit printes en fejlbesked i stedet.

\subsubsection{Funktioner til behandling af \emph{database\_t} typen}
\paragraph{void addStudent( database\_t* database, student\_t* student )}
Hvis fejl-flaget ikke er sat, tilføjes den studerende til databasen; hvis det er
printes en besked herom. I begge tilfælde printes dataen i den modtagende
studerende til terminalen. Den studerende tilføjes naturligvis kun, hvis der er
plads til denne i databasen, ellers printes en fejlbesked om at databasen er
fuld.

\paragraph{void listStudents( database\_t* database )} Går igennem alle de
studerende og kalder \emph{void printStudent( student\_t* student )} på dem.
Forud for dette printes den studerendes nummer (placering i databasen).

Desuden udregnes alle de studerendes gennemsnit. Studerende med korrupt data
medtages ikke i denne beregning.

\subsubsection{Brugerflade}
I \emph{main} funktionen befinder sig en løkke, der leder efter inputs fra
brugeren. Før hvert kald efter bruger input kaldes \emph{fflush( stdout )} da
eclipsekonsollen ikke selv håndterer at tømme bufferen efter hver linje.
Ligeledes kaldes \emph{rewind( stdin )} efterfølgende, for at fjerne eventuelle
linjeskift, der ville blive opfattet ved næste kald efter input.

Indtastes andet end tal, smides en besked om at det er uden virkning og
indtastes et tal, der ikke har nogen funktion, mindes brugeren om, hvad de
forskellige tal gør.

\paragraph{Testfunktioner} Der findes to muligheder for at teste databasen i
``menuen'' den ene $3$ opretter en ny database og tilføjer nogle personer til
denne, for at teste at forkerte inpust afvises. 
Desuden udføres en test af fyldt database ved at lægge 10000 ens studerende ind.
Da dette fylder konsollen ventes der på et brugerinput, der dog ikke benyttes
som andet end startbesked. Mere om test senere.

Den anden $4$ korrupterer dataen for en valgt studerende, så paritetskontrollen
kan testes.

\subsection{Test}
Testmetoden kales fra terminalen med kommandoen $3$. Denne frsøger at tilføje 13
studerende til en ny database. Der tilføjes studerende med start år, der er både
legale og inkorrekte dataindtastniger.

For hver test studerende står først den indtastede data. Det bør bemærkes at
dette ikke er data indtastet fra konsollen, men kald til \emph{encodeStudent}
efterfulgt af \emph{addStudent}.

Alt opfører sig som forventet, omend med to særtilfælde:
\paragraph{J\\0 12abc} Tilføjes, selvom navnet er for langt. Der stoppes dog med
at læse fra navnet når escapekarakteren ``\\0'' nåes. Dette gør at den
studerende blot får navnet ``J''. I helle arrayet står: $[J,\\0,\\0,\\0,\\0]$

\paragraph{J\\nOH} Tilføjes ligeledes, navnet er jo på fire karakter, men det
gliver nogle ``grimheder'', når de studerende skal vises, da linjeskiftet
printes. Det ville ikke være svært at frasortere denne type af navne, men dette
var ikke en prioitet.

Gennemsnittet udregnes også korrekt til 36.43 da $\frac{255}{}7 \approx
36.42857$

Dernæst tilføjes 10000 ens studerende. Des ses at databasen ikke længere
modtager studerende efter s9999, altså den 10000. studerende. Som ønsket

Hele testsekvensen findes her:
\begin{lstlisting}[caption=Kald til testfunktionen\, der er klippet
ca. 10000 linjer ud\, da disse følger samme monster]
Welcome to CUDB - The C University
Data Base

0: Halt
1: List all students
2: Add a new student
3: Test database
4: Corrupt student data >:) (use only for testing)

Enter command: 
3
Initializing test:
( "JOH0", 2009, 0, 0 )
Adding student: s0000 JOH0 2009 Autumn 000
( "JOH1", 2008, 0, 0 )
ERROR! Student data is illegal! JOH1 2009 Autumn 000
( "JOH2", 2040, 0, 0 )
Adding student: s0001 JOH2 2040 Autumn 000
( "JOH3", 2041, 0, 0 )
ERROR! Student data is illegal! JOH3 2009 Autumn 000
( "JOH4", 2009, 1, 0 )
Adding student: s0002 JOH4 2009 Spring 000
( "JOH5", 2009, 2, 0 )
ERROR! Student data is illegal! JOH5 2009 Autumn 000
( "JOH6", 2009, -1, 0 )
ERROR! Student data is illegal! JOH6 2009 Autumn 000
( "JOH7", 2009, 0, 255 )
Adding student: s0003 JOH7 2009 Autumn 255
( "JOH8", 2009, 0, 256 )
ERROR! Student data is illegal! JOH8 2009 Autumn 000
( "JOH9", 2009, 0, -1 )
ERROR! Student data is illegal! JOH9 2009 Autumn 000
( "JOH10", 2009, 0, 0 )
ERROR! Student data is illegal!  2009 Autumn 000
( "J11", 2009, 0, 0 )
Adding student: s0004 J11 2009 Autumn 000
( "J\0 12abc", 2009, 0, 0 )
Adding student: s0005 J 2009 Autumn 000
( "J\nOH", 2009, 0, 0 )
Adding student: s0006 J
OH 2009 Autumn 000
s0000 JOH0 2009 Autumn 000
s0001 JOH2 2040 Autumn 000
s0002 JOH4 2009 Spring 000
s0003 JOH7 2009 Autumn 255
s0004 J11 2009 Autumn 000
s0005 J 2009 Autumn 000
s0006 J
OH 2009 Autumn 000

Average grade: 36.43
Write something to do test of filled database
3
Adding 10000 students
Adding student: s0007 Many 2009 Autumn 128
Adding student: s0008 Many 2009 Autumn 128
...
...
Adding student: s9995 Many 2009 Autumn 128
Adding student: s9996 Many 2009 Autumn 128
Adding student: s9997 Many 2009 Autumn 128
Adding student: s9998 Many 2009 Autumn 128
Adding student: s9999 Many 2009 Autumn 128
Database is full, contact your local SYS-admin (or don't)
Database is full, contact your local SYS-admin (or don't)
Database is full, contact your local SYS-admin (or don't)
Database is full, contact your local SYS-admin (or don't)
Database is full, contact your local SYS-admin (or don't)
Database is full, contact your local SYS-admin (or don't)
Database is full, contact your local SYS-admin (or don't)

Enter command:
0
Bye!
\end{lstlisting}

Brugerfladen testes nu:
Først testes navne der er for lange. Det ses at disse slipper igennem, men kun
de første fire bogstaver gives videre, ellers ville \emph{encodeStudent} fejle:
\begin{lstlisting}[caption=Test af for lange navne]
Enter command: 
2
Name, 4 chars: John Smith
Year, 2009-2040: 2010
Semester, 0/1: 1
Grade, 0-255: 34
Adding student: s0000 John 2010 Spring 034
\end{lstlisting}

Det testes hvad der sker når tekst indtastes i stedet for tal:
\begin{lstlisting}[caption=Test af tekst i talfelt]
Enter command: 
2
Name, 4 chars: Derp
Year, 2009-2040: Fo
Semester, 0/1: 0
Grade, 0-255: 128
Error in entered data
\end{lstlisting}
Ligeledes for tal, der ikke er ``tiladte''. Det ses at feltet i så fald bliver
nulstillet og en fejlbesked printes:
\begin{lstlisting}[caption=Test af forkerte tal i talfelt]
Enter command: 
2
Name, 4 chars: Herp
Year, 2009-2040: 2020
Semester, 0/1: 1
Grade, 0-255: 1337
ERROR! Student data is illegal! Herp 2020 Spring 000
\end{lstlisting}
Disse to håndteres forskelligt, da tekst i et talfelt ikke giver noget tal at
vidderegive til \emph{encodeStudent} og fejlen derfor smides af
UI-håndteringsløkken. Testene udføres ikke for resten af felterne, da disse er
kodet på samme måde.

Der tilføjes et par legale studerende:
\begin{lstlisting}[caption=Legale studerende]
Enter command: 
2
Name, 4 chars: Jane
Year, 2009-2040: 2018
Semester, 0/1: 0
Grade, 0-255: 234
Adding student: s0001 Jane 2018 Autumn 234

Enter command: 
2
Name, 4 chars: Jack
Year, 2009-2040: 2040
Semester, 0/1: 1
Grade, 0-255: 12
Adding student: s0002 Jack 2040 Spring 012
\end{lstlisting}

Og deres samlede gennemsnit udregnes:
\begin{lstlisting}[caption=Print af database]
Enter command: 
1
s0000 John 2010 Spring 034
s0001 Jane 2018 Autumn 234
s0002 Jack 2040 Spring 012

Average grade: 93.33
\end{lstlisting}

Jane korrupteres:
\begin{lstlisting}[caption=Korruptering af Jane]
Enter command: 
4
Student id to corrupt: 1
Before: Jane 2018 Autumn 234
After: ERROR! DATA IS CORRUPTED!!
\end{lstlisting}

Databasen printes igen, det ses at Jane ikke tæller med i gennemsnitsudregningen
mere:
\begin{lstlisting}[caption=Print af korrupt database]
Enter command: 
1
s0000 John 2010 Spring 034
s0001 ERROR! DATA IS CORRUPTED!!
s0002 Jack 2040 Spring 012

Average grade: 23.00
\end{lstlisting}

Her har jeg glemt, hvordan man slutter programmet og prøver at skrive ``hjælp''.
Dette håndteres også korrekt:
\begin{lstlisting}[caption=Tekst i kommandofelt]
Enter command: 
Hjælp
Only numbers are accepted
\end{lstlisting}

Derefter indtastes ``9001'', for at tjekke håndtering af tal-inpust, der ikke
har nogen kommando:
\begin{lstlisting}[caption=Kommandol\oe dst tal i kommandofelt]
Enter command: 
9001
Unknown command. Try:

0: Halt
1: List all students
2: Add a new student
3: Test database
4: Corrupt student data >:) (use only for testing)
\end{lstlisting}

Programmet lukkes ved at indtaste ``0'':
\begin{lstlisting}[caption=afslutnign af program]
Enter command: 
0
bye!
\end{lstlisting}

\subsubsection{Fejl}
Der er fundet tre eksempler på ikke optimal opførsel.
Den ene er, hvis alt for store, eller små tal indtastes, sker der uforudsigelige
ting
\begin{lstlisting}[caption=Afslutnign af program]
Enter command: 
Enter command: 
-9999999999999999999999999999999999999999999

Average grade: -1.#J
\end{lstlisting}
Dette skyldes sandsynligvis overflow.
At gennemsnittet udregnes til $-1.\#J$ skyldes at der ingen studerende er i
databasen og der derfor deles med nul, dette er fuldt legalt i C. $-1.\#J$
betyder sandsyligvis enten $-\infty$, $\infty$ eller deling med $0$.
Dette ses dog ikke som en alvorlig fejl, da et gennemsnit for ingen studerende
jo ikke er definerbart.

den sidste ``fejl'' opstår, hvis programmet termineres direkte fra
\emph{Eclipse}, dette medfører en opførsel der tyder på at en masse tomme linjer
sendes til konsollen:
\begin{lstlisting}[caption=Terminering af program fra Eclipse]
Enter command: 
Only numbers are accepted
...
...
Enter command: 
Only numbers are accepted

Enter command: 
Only numbers are accepted

Enter command: 
Only numbers are accepted

Enter command: 
Only numbers are accepted
\end{lstlisting}
Antalet af linjer varierer, men sm regel mellem 12 og 14.


