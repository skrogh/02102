\newcommand{\gol}{\emph{GameOfLife}}

\section{Opgave 1}

\subsection{Specifikation}
Der ønskes et program, der kan udfører simuleringer af celle liv efter modellen \emph{Conway's Game of Life},
se evt \url{http://en.wikipedia.org/wiki/Game_of_life}.

følgende er implimenteret:
\begin{itemize}
  \item et \emph{GameOfLife} objekt, der dog også kan håndterer andre cellulare automier.
  \item Constructor der tillader at initalisere et \emph{GameOfLife} objekt
  med en given størrelse, tilfældigt eller ej.
  \item Constructor der tillader at initalisere et \emph{GameOfLife} objekt
  med en given start tilstand.
  \item Mulighed for at tage et enkelt skridt frem.
  \item Mulighed for at tage skridt kontinuert.
  \item Mulighed for at hente \emph{GameOfLife} objekter fra filer.
  \item Mulighed for at hente andre regelsæt end game of life fra filer.
  \item Mulighed for at hente farve fra filer.
  \item Mulighed for at manipulerer \emph{GameOfLife} objekter med musen.
  \item ``Banen'' er sat op som en torus, dvs. kanterne rører hinanden.
\end{itemize}

\subsection{Design}

\subsubsection{ GameOfLifeMain klassen }
Dette er en statisk klasse, der lader en bruger lege med \emph{GameOfLife} klassen.
Brugeren kan lave og manipulerer et \emph{GameOfLife} objekt her.

Et dummy \emph{GameOfLife} objekt tegnes på et $512\times512$ canvas og
brugeren bedes om at vælge mellem at hente et sæt filer, lave en tilfældig bane eller en ``ren'' bane.

Brugeren kan nu simulerer enkelte skridt ved at trykke på \emph{mellemrum},
eller flere ved at holde tasten.

Brugeren kan også manipulerer celler med musen.

\paragraph{ \gol initialsering }
Brugeren bedes om at vælge mellem 3. muligheder.

Skrives ordet \emph{random}, spørges om en størrelse og et \gol objekt laves med denne størrelse, tilfældig farve og starttilstand.

Skrives ordet \emph{clean}, spørges om en størrelse og et \gol objekt laves med denne størrelse, tilfældig farve og alle celler som døde.

Skrives ordet \emph{load}, spørges om en sæt filer.
Regelfil, starttilstandsfil og farvefil. Både regel- og farvefilen accepterer ``N'';
I hvilket tilfælde Conway's Game of Life hhv. tilfældige farver benyttes.

\paragraph{ Skridt }
Der tages et skridt, hver gang der trykkes på mellemrum. Holdes denne inde skriver maskinen naturligvis flere til key-bufferen.
Denne tømmes, hvert skridt - da hvert skrict kan tage lang tid at udregne og simulieringen ikke ville stoppe, når mellemrumstasten løftes da.

\paragraph{ Mussemanipulation }
Trykkes der med musen, gemmes værdien af cellen under den. Så længe den holdes, sættes celler under den til denne værdi plus én.
\gol objektet holder styr på at ``folde'' værdier, så en levende celle bliver til en død, i stedet for en celle uden regler.


\subsubsection{ GameOfLife klassen}

\gol klassen har følgende data:
\begin{itemize}
  \item int edgeState; Bestemmer hvilen tilstand sidderne har. -1 for en torus.
  \item int[][] state; Gemmer data om værdien i alle celler. Dette array forudsættes at være med lige kanter, og ikke tomt.
  \item int[][] rules; Gemmer regelsættet
  \item int states; Antallet af forskellige states, det samme som rækkeantallet i rules.
  \item Color[] colors; Array af farver til cellerne.
\end{itemize}




\paragraph{ GameOfLife(int, int, boolean)} Laver et \gol objekt med Conway's regler og en given størrelse. Der kan vælges en tilfældig starttilstand.

\paragraph{ GameOfLife(int[][]) } Laver et \gol obejkt med Conway's regler og en given starttilstand. Er denne ikke valid laves et \gol med én celle.

\paragraph{GameOfLife()} aver et \gol obejkt med Conway's regler og én celle.

\paragraph{setRules(int[][])} Sætter et regelsæt, hvis dette er gyldigt og ingen celler lever, der ikke findes i regelsættet.
I et regelsæt er hver række regelsættet for en celletilstand, første kolonne angiver antalet af ``liv'' denne tælder for,
mere om dette under \emph{step()}, de næste kolonner angiver, hvad der sker med denne celle, når summen af ``liv''
omkring den har forskellige værdier. første kolonne er ved 0 eller mindre, næste 1, så 2 osv. et større antal liv ender altid i kolonnen længst til højre.

kalder \emph{addMissingColors()}

\paragraph{setRules()} Sætter regelsættet til \emph{Conway's Game of Life}

\paragraph{addMissingColors()} Tilføjer farver til alle celletyper uden farver.

\paragraph{neighbors(int, int)} Udregner antallet af naboliv for en given celle, dette gøres ved at kigge på cellerne rundt om.
Dem under, dem ved siden af og dem over. 
Er \emph{edgeState}=-1 kaldes \emph{mod} på begge komposanter for koordinatet til celler rundt om,
dette gør at ex celle$(10, 5)$ med state størrelse $(10, 10)$ bliver til celle$(0, 5)$

\paragraph{step()} Udregner og 


\paragraph{render(double, double, double, double)}
\paragraph{renderMouse(double, double, double, double)}
\paragraph{mouseX(double, double, double, double)}
\paragraph{mouseY(double, double, double, double)}
\paragraph{toString()}
\paragraph{printNeighbors()}
\paragraph{mod(int, int)}
\paragraph{alterState(int, int)}
\paragraph{setState(int, int, int)}
\paragraph{setStates(int[][])}
\paragraph{getRules()}
\paragraph{getColors()}
\paragraph{getStates()}
\paragraph{getState()}
\paragraph{getState(int, int)}
\paragraph{validRules(int[][])}
\paragraph{setColors(Color[])}
\paragraph{loadSetup(String, String, String, boolean)}
\paragraph{loadSetup(int[][], int[][], Color[], boolean)}
\paragraph{loadSetup(int[][], Color[], boolean)}
\paragraph{loadSetup(int[][], boolean)}
\paragraph{randomize()}